\chapter{Introduction}
The objective of the project is to discover disease subtypes of Prostate adenocarcinoma from a dataset coming from The Cancer Genome Atlas (TCGA)\cite{hutter2018TCGA}. For the examined cancer we take into consideration the dataset with code "\textit{PRAD}". In particular we will consider the subtypes identified by The Cancer Genome Atlas Research Network\cite{abeshouse2015molecularPRAD}. We will be looking at a subset of the types of data analysed in\cite{abeshouse2015molecularPRAD}, namely the biomolecules  obtained from mRNA and miRNA sequencing, and reverse-phase protein array of the samples. The Cancer Genome Atlas Research Network performed unsupervised
clustering of data from each molecular platform, as well as integrative clustering using iCluster\cite{shen2009integrative}. The latter approach is more ideal seen that it allows joint inference from multi-omics data by generating a single integrated cluster assignment through simultaneously capturing patterns of genomic alterations. Instead, identifying tumor subtypes by separately clustering each data view requires to manually integrate the results.
The analyses with both individual and integrative clustering of the TCGA Prostate adenocarcinoma dataset uncovered both known and novel associations, with 74\% of all tumors being assignable to one of seven molecular classes. These are based on distinct oncogenic drivers: fusions involving (1) ERG, (2) ETV1, (3) ETV4, or (4) FLI1; mutations in (5) SPOP
or (6) FOXA1; or (7) IDH1 mutations. Thus, more generally, the tumors can be subdivided into 3 disease subtypes. We will try to predict the membership of a sample in the multi-omics dataset to a class by applying different clustering algorithms.

